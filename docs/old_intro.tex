\section{Introduzione}\label{sec:Introduzione}
Questo elaborato rappresenta la relazione del progetto svolto per l'esame di Ingegneria del Software del corso di Laurea Triennali in Ingegneria Informatica dell'Università degli Studi di Firenze.
Il codice sorgente del progetto è disponibile su \textit{GitHub} nel seguente indirizzo: \href{https://github.com/TommyAen/SWE-Project}{SWE-Project}.
\subsection{Statement}
Questo progetto si propone di sviluppare un'applicazione per la gestione e l'utilizzo di servizi di carsharing dedicato agli studenti universitari: l'università mette a disposizione delle auto (navette) che possono essere usate dagli studenti per spostarsi tra le sedi universitarie in modo efficiente e semplice.
\newline
Il sistema deve poter memorizzare i dati anagrafici degli utenti, che verranno forniti in fase di registrazione con possibilità di modifica in futuro.
Se uno studente è in possesso di patente B, allora può dare la disponibilità a essere guidatore, mentre studenti senza patente potranno solo prenotarsi come passeggeri.
\newline
\textit{La prenotazione di un posto può essere effettuata entro la sera precedente al giorno del viaggio.}
\newline
Il gestore deve poter modificare o eliminare le prenotazioni, \textit{notificando gli utenti tramite email}, e deve anche poter gestire i profili utente, revocando il permesso di guida in caso di patente non più valida o eliminando il profilo utente se necessario. 
Deve avere anche la possibilità di aggiungere o rimuovere navette nelle varie sedi in caso di nuove disponibilità o malfunzionamenti.

\subsection{Architettura e pratiche utilizzate}
Il software è stato sviluppato in Java, con il supporto di un database PostgreSQL per la memorizzazione e la gestione persistente dei dati. Per la connessione tra l’applicazione e il database è stata utilizzata la libreria JDBC (Java DataBase Connectivity).\\
\noindent
Per garantire una chiara separazione delle responsabilità, l’architettura del progetto è stata suddivisa in tre macro-componenti principali: Domain Model, Business Logic e ORM. Ciascun package ha un ruolo ben definito nella struttura complessiva del sistema.
\begin{itemize}
\item \textbf{Domain Model}: definisce le entità principali dell’applicazione, tra cui \textit{Utente}, \textit{Prenotazione}, \textit{Tratta}, ecc.
\item \textbf{Business Logic}: contiene le classi che implementano le funzionalità principali del sistema, come la gestione delle prenotazioni, il controllo dei permessi e la logica di matching tra utenti.
\item \textbf{ORM}: include le classi responsabili del mapping tra oggetti Java e tabelle del database, permettendo la lettura e scrittura dei dati in modo astratto e modulare.
\end{itemize}
I diagrammi UML, inclusi gli Use Case Diagram e i Class Diagram, sono stati realizzati seguendo lo standard UML (Unified Modeling Language).\\
\noindent Di seguito l’elenco degli strumenti e delle piattaforme utilizzati durante lo sviluppo:
\begin{itemize}
\item \textbf{Visual Studio Code}: IDE utilizzato per lo sviluppo in Java
\item \textbf{PgAdmin}: interfaccia grafica per la gestione di database PostgreSQL
\item \textbf{GitHub}: piattaforma per il versionamento e la condivisione del codice sorgente
\item \textbf{draw.io}: utilizzato per la realizzazione del navigation diagram e dei diagrammi UML
\item \textbf{Figma}: software per la creazione di mock-up
\item \textbf{Overleaf.com}: piattaforma per la stesura collaborativa del report in \LaTeX
\end{itemize}

\newpage