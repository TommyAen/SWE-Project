\section{Testing}

Per verificare il corretto funzionamento del progetto abbiamo suddiviso i test due categorie principali:
\begin{itemize}
    \item Test ORM: Verificano la corretta persistenza dei dati nel database attraverso le classi DAO.
    \item Test di Business Logic: Verificano il corretto funzionamento dei controller che gestiscono la logica di business del sistema.
\end{itemize}
Abbiamo deciso di non fare test su Domain Model in quanto contiene classi composte esclusivamente attributi e metodi getter-setter, senza logica complessa da testare, e non presentano dipendenze esterne.\\
I test sono stati sviluppati con JUnit 5, e ciascuna classe di test si occupa di ripulire i dati inseriti nel database al termine dell'esecuzione, garantendo così l'indipendenza tra i test.\\
Negli unit test è stato usato il pattern di \textit{Dependency Injection} per iniettare le dipendenze necessarie alle classi da testare, facilitando così l'isolamento delle unità di codice.\\
\begin{lstlisting}[style=java, caption={Esempio di Dependency Injection in UserControllerTest}]
@BeforeEach
void setUp() {
    userDAO = new UserDAO();
    authController = new AuthController(userDAO);
    userController = new UserController(userDAO, authController);
    // ... creazione di un oggetto User per i test ...
}
@Test
void testRegister() throws SQLException {
    userController.register(1, "Prova", "Register", "prova.register@example.com", "password123", "B12345");
    User registeredUser = userDAO.findById(1); // recupero dell'utente registrato dal database
    assertNotNull(registeredUser);
    assertEquals("Prova", registeredUser.getName());
    authController.loginById(1, "password123");
    assertTrue(authController.isLoggedIn());
}
\end{lstlisting}
\subsection{Test riguardanti casi d'uso}
Per ogni caso d'uso descritto nella sezione \ref{subsec:usecaseTemplate} sono stati implementati test specifici per verificare il corretto funzionamento delle funzionalità principali del sistema: di sotto viene riportato il mapping tra i test fatti e la descrizione dei casi d'uso.
\begin{itemize}
    \item \texttt{testRegister} → UC-1
    \item \texttt{testViewAvailableTrips} → UC-2
    \item \texttt{testJoinTrip} → UC-3
    \item \texttt{testCreateNewTrip} → UC-4
    \item \texttt{testDeleteUser} → UC-5
    \item \texttt{testAddVehicle} → UC-6
\end{itemize}
\subsection{Elenco unit test implementati}
\paragraph{Test ORM}
Di seguito sono elencati i principali test implementati per verificare il corretto funzionamento delle classi DAO:
\begin{itemize}
    \item \texttt{UserDAOTest}: 
    \begin{itemize}
        \item \texttt{testInsertUser()}
        \item \texttt{testUpdateUserEmail()} 
        \item \texttt{testFindUserByID()}
        \item \texttt{testAddLicense()}
        \item \texttt{testRemoveLicense()}
        \item \texttt{testRemoveUserByID()}.
    \end{itemize}
    
    
    \item \texttt{TripDAOTest}: 
    \begin{itemize}
        \item \texttt{testInsertTrip()}
        \item \texttt{testFindById()}
        \item \texttt{testUpdateTripDate()}.
    \end{itemize}
    
    
    \item \texttt{LocationDAOTest}: 
    \begin{itemize}
        \item \texttt{testAddLocation()}
        \item \texttt{testFindByName()}
        \item \texttt{testUpdateCapacity()}
        \item \texttt{testUpdateCapacityNegative()}
        \item \texttt{testRemoveLocation()}
    \end{itemize}
    
    \item \texttt{VehicleDAOTest}: 
    \begin{itemize}
        \item \texttt{testInsertVehicle()}
        \item \texttt{testFindInLocation()}
        \item \texttt{testFindAvailableInLocation()}
        \item \texttt{testUpdateStatus()}
        \item \texttt{testRemoveVehicle()}
    \end{itemize}
\end{itemize}

\paragraph{Business Logic Test} \label{subsec:business-logic-test}
Per rendere indipendenti i test di Business Logic dai DAO, è stato usato il mocking tramite \texttt{Mockito}, che simulano il comportamento di funzionalità delle classi reali, restituendo dati predefiniti.\\
\begin{lstlisting}[style=java, caption={Esempio uso di mocking  in BookingControllerTest.}, label={lst:bookDAOmocking}]
    @Mock
    private BookingDAO bookingDAO;
    @Mock
    private TripDAO tripDAO;
    @Mock
    private AuthController authController;
    @Mock
    private TripController tripController;
    @InjectMocks
    private BookingController bookingController;
    
    private Trip testTrip;
    void setup() {
    // .. Setup di un qualche oggetto trip per i test ..
        testTrip = //...
    }
    
    void testBookTripFull() throws SQLException {
        when(tripController.isFull(testTrip)).thenReturn(true); //simula che viaggio sia pieno
        
        Booking newBooking = bookingController.createBooking(testTrip.getId());
        
        verify(bookingDAO, never()).insertBooking(any()); //verifica che insertBooking non venga mai chiamato
        assertNull(newBooking); //verifica che il risultato sia null
    }
\end{lstlisting}
Come si può vedere nel listato \ref{lst:bookDAOmocking}, l'annotazione \texttt{@Mock} viene utilizzata per creare oggetti mock delle classi DAO e del controller di autenticazione, creando automaticamente degli oggetti \texttt{mock} per quei tipi, mentre \texttt{@InjectMocks} viene usata per iniettare questi mock nel controller che si vuole testare, in questo caso \texttt{BookingController}.\\
Di seguito sono elencati i principali test implementati per verificare il corretto funzionamento dei controller:
\begin{itemize}
    \item \texttt{UserControllerTest}:
    \begin{itemize}
        \item \texttt{registerStudent()}
        \item \texttt{testDeleteThisProfile()}
        \item \texttt{testDeleteProfile()}
        \item \texttt{testLogin()}
        \item \texttt{testLoginFail()}.
    \end{itemize}
    \item \texttt{TripControllerTest}: 
    \begin{itemize}
        \item \texttt{testCreateTrip()}
        \item \texttt{testFindById()}
        \item \texttt{testModifyTrip()}
        \item \texttt{testCancelTrip()}
    \end{itemize}
    \item \texttt{BookingControllerTest}
    \begin{itemize}
        \item \texttt{testBookAndCancel()}
        \item \texttt{testBookFullTrip()}
    \end{itemize} 
    \item \texttt{LocationControllerTest}
    \begin{itemize}
        \item \texttt{testAddLocation()}
        \item \texttt{testAddLocationNotAdmin()}
        \item \texttt{testFindById()}
        \item \texttt{testUpdateCapacity()}
    \end{itemize}
    \item \texttt{VehicleControllerTest}
    \begin{itemize}
        \item \texttt{testAddVehicle()}
        \item \texttt{testAddVehicleNotAdmin()}
        \item \texttt{testFindById()}
        \item \texttt{testModifyStatus()}
        \item \texttt{testListAvailableVehiclesForLocation()}
    \end{itemize}
\end{itemize}