\section{Testing}

Per verificare il corretto funzionamento del progettoI test sono stati suddivisi in due categorie principali:
\begin{itemize}
    \item Test di Business Logic: Verificano il corretto funzionamento dei controller che gestiscono la logica di business del sistema.
    \item Test ORM: Verificano la corretta persistenza dei dati nel database attraverso le classi DAO.
\end{itemize}
Abbiamo deciso di non fare test su Domain Model in quanto contiene classi che contengono solo attributi e metodi getter-setter, senza logica complessa da testare, e non presentano dipendenze esterne.\\
I test sono stati sviluppati con JUnit 5, e ciascuna classe di test si occupa di ripulire i dati inseriti nel database al termine dell'esecuzione, garantendo così l'indipendenza tra i test.\\
Negli unit test è stato usato il pattern di \textit{Dependency Injection} per iniettare le dipendenze necessarie alle classi da testare, facilitando così l'isolamento delle unità di codice.\\
\begin{lstlisting}[style=java, caption={Esempio di Dependency Injection in UserControllerTest}]
@BeforeEach
void setUp() {
    userDAO = new UserDAO();
    authController = new AuthController(userDAO);
    userController = new UserController(userDAO, authController);
    // creazione di un oggetto User per i test
}
@Test
void testRegister() throws SQLException {
    userController.register(1, "Prova", "Register", "prova.register@example.com", "password123", "B12345");
    User registeredUser = userDAO.findById(1); // recupero dell'utente registrato dal database
    assertNotNull(registeredUser);
    assertEquals("Prova", registeredUser.getName());
    authController.loginById(1, "password123");
    assertTrue(authController.isLoggedIn());
}
\end{lstlisting}
\subsection{Test riguardanti casi d'uso}
Per ogni caso d'uso descritto nella sezione \ref{subsec:usecaseTemplate} sono stati implementati test specifici per verificare il corretto funzionamento delle funzionalità principali del sistema: di sotto viene riportato il mapping tra i test fatti e la descrizione dei casi d'uso.
\begin{itemize}
    \item \texttt{testRegister} → UC-1
    \item \texttt{testViewAvailableTrips} → UC-2
    \item \texttt{testJoinTrip} → UC-3
    \item \texttt{testCreateNewTrip} → UC-4
    \item \texttt{testDeleteUser} → UC-5
    \item \texttt{testAddVehicle} → UC-6
\end{itemize}
\subsection{Elenco dei unit test implementati}


